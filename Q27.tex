\ProvidesFile{Q27.tex}[Билет 27]

\section{Билет 27. Сформулируйте формулу Грина и докажите её для области $\Omega \subset \mathbb{R}^2$ вида $\Omega = [a, b] \times [c, d]$ (прямоугольник).}

\begin{theorem}{Формула Грина.}\\
    Пусть $U \subset \mathbb{R}^2$ -- связное подмножество, ограниченное кусочно-гладкой кривой $\partial U = \Gamma$ (граница множества $U$).
    Зафиксируем ориентацию на $\Gamma$, обход вдоль которой всегда оставляет область $U$ слева. Пусть функции $P(x, y), Q(x, y)$ дифференцируемы
    в некоторой окрестности $U$. Тогда верно следующее:
    \begin{enumerate}
        \item \[ \int\limits_{\Gamma} P(x, y)dx = \iint\limits_{U} -\frac{\partial P}{\partial y}dxdy \]
        \item \[ \int\limits_{\Gamma} Q(x, y)dy = \iint\limits_{U} \frac{\partial Q}{\partial x}dxdy \]
        \item \[ \int\limits_{\Gamma} Pdx + Qdy =
                \iint\limits_{U} \left(\frac{\partial Q}{\partial x} - \frac{\partial P}{\partial y}\right)dxdy \text{ -- формула Грина}\]
    \end{enumerate}
\end{theorem}

\begin{proof}
    План:
    \begin{enumerate}
        \item Докажем формулу 1.
        \item Формула 2 доказывается аналогично при помощи замены $x \leftrightarrow y$.
        \item Формула 3 суть сумма формул 1 и 2.
    \end{enumerate}
    Будем предполагать, что $U = \left\{ (x, y) \, | a \leqslant x \leqslant b, g_1(x) \leqslant y \leqslant g_2(x) \right\}$,
    т.е. $U$ -- криволинейная трапеция.\\
    $\Gamma = \partial U = \Gamma_1 \cup \Gamma_2 \cup \Gamma_3 \cup \Gamma_4$, где $\Gamma_1, \Gamma_2$ -- графики $g_1$ и $g_2$,
    а $\Gamma_3, \Gamma_4$ -- вертикальные отрезки. Тогда:
    \[
        \int\limits_{\Gamma} Pdx =
        \int\limits_{\Gamma_1} Pdx + \int\limits_{\Gamma_2} Pdx + \int\limits_{\Gamma_3} Pdx + \int\limits_{\Gamma_4} Pdx
    \]
    Заметим, что $\int\limits_{\Gamma_3} Pdx + \int\limits_{\Gamma_4} Pdx = 0$,
    т.к. на вертикальных отрезках выполнено, что $x = const \Rightarrow dx = 0$.\\
    Параметризуем: $\Gamma_1 = \begin{cases} x = t\\ y = g_1(t)\end{cases} t \in [a, b]$, $\Gamma_2 = \begin{cases} x = t\\ y = g_2(t)\end{cases} t \in [b, a]$.
    Интегрируем по определению:
    \[
        \int\limits_{a}^{b} P(t, g_1(t))dt - \int\limits_{a}^{b} P(t, g_2(t))dt =
        \int\limits_{a}^{b} \left[P(t, g_1(t)) - P(t, g_2(t)) \right] dt = \star
    \]
    Заметим, что подынтегральное выражение равно (по формуле Ньютона-Лейбница):
    \[
        \int\limits_{g_2(t)}^{g_1(t)} \frac{\partial P}{\partial y}(t, s)ds =
        -\int\limits_{g_1(t)}^{g_2(t)} \frac{\partial P}{\partial y}(t, s)ds
    \]
    Подставим:
    \[
        \star = -\int\limits_a^b \left[\int\limits_{g_1(t)}^{g_2(t)} \frac{\partial P}{\partial y}(t, s)ds \right]dt = [\text{Переходим к кратному интегралу}] =
        -\iint\limits_{U} \frac{\partial P}{\partial y}dxdy
    \]
\end{proof}