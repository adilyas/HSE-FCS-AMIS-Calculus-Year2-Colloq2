\ProvidesFile{Q26.tex}[Билет 26]

\section{Билет 26. Объясните, почему криволинейный интеграл 1-го рода не зависит от ориентации кривой, а криволинейный
интеграл 2-го рода -- зависит.}
\begin{definition}
    Предел $\lim\limits_{|\tau| \to 0} S_{\tau, f, \gamma}^{I} = \int\limits_{\gamma} f(x_1, ..., x_n)dl$ называется
    криволинейным интегралом 1-го рода от функции $f$ по кривой $\gamma$.
\end{definition}

\begin{remark}
    $dl$ называется элементом длины кривой. По теореме Пифагора:
    \[
        dl = \sqrt{(dx_1)^2 + ... + (dx_n)^2} = \left[x_j = \varphi_j(t)\right] =
        \sqrt{(\varphi_1^{'}(t)dt)^2 + ... + (\varphi_n^{'}(t)dt)^2} =
        \sqrt{\varphi_1^{'}(t)^2 + ... + \varphi_n^{'}(t)^2}dt
    \]
\end{remark}

\begin{definition}
    Предел $\lim\limits_{|\tau| \to 0} S_{\tau, f, \gamma}^{II} = \int\limits_{\gamma} f(x_1, ..., x_n)dx_1$ называется
    криволинейным интегралом 2-го рода от функции $f$ по кривой $\gamma$ по переменной $x_1$.
\end{definition}

\begin{remark}
    $dx_1 = d\varphi_1(t) = \varphi_1^{'}(t)dt$
\end{remark}

При изменении направления обхода интеграл 2-го рода меняет знак, а интеграл 1-го рода не меняет знак. Почему?
Заметим следующее: в интегралах 2-го рода стоят выражения $dx_1, ..., dx_n$, которые меняют знак при изменении направления обхода,
а элемент длины $dl = \sqrt{(dx_1)^2 + ... + (dx_n)^2}$ не зависит от знаков выражений $dx_1, ..., dx_n$.

\begin{remark} Подробнее про смену знака $dx_i$ (из билета 25).\\
    Определим интегральную сумму (2-го рода) по кривой $\gamma$:
    \[
        S_{\tau, f, \gamma}^{II, \, x_1} = \sum_{i = 1}^{m} f\left[\varphi(c_i)\right] \Delta x_1
    \]
    \[
        \Delta x_1 = \varphi_1(t_i) - \varphi_1(t_{i - 1})
    \]
    Видим, что $\Delta x_i$ в определении интегральной суммы зависит от порядка обхода сегментов разбиения.
\end{remark}