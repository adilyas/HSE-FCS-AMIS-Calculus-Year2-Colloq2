\documentclass{article}
\usepackage{packages}

\title{Коллоквиум по Математическому анализу-2, семестр 2}
\author{Виноградова Дарья, Залялов Александр, Миронов Алексей, Стрельцов Артём, Т}
\date{}

\begin{document}

	\maketitle

	\tableofcontents

    \clearpage
    
    \section*{Реклама}
    \href{https://t.me/applied_memes}{@applied\_memes}\par
    \href{https://t.me/fcs_channels}{@fcs\_channels}

	\setcounter{section}{13}

	\section{Дайте определение гладкого $k$-мерного подмногообразия в $\mathbb{R}^n$ и сопутствующее определение гладких координат. Приведите пример параметрической кривой, которая параметрически задана дифференцируемыми функциями, но не является гладким 1-мерным многообразием в какой-нибудь точке}
\begin{definition}
	Подмножество $M\subseteq \mathbb{R}^n$ называется \textit{гладким $k$-мерным (под)многообразием} в $\mathbb{R}^n$, если $\forall x \in M$ существует окрестность $U$, $x\in U$, такая что на $M \cap U$ можно задать гладкие координаты.
\end{definition}
\begin{definition}
\textit{Гладкие координаты} --- отображение $\Phi: V \xrightarrow{} M$, где $V \subseteq \mathbb{R}^k$, задаваемое уравнениями
\end{definition}
\begin{gather*}
\begin{cases}
    x_1 = \phi_1(t_1, \dotsc, t_k)\\
    \vdots\\
    x_n = \phi_n(t_1, \dotsc, t_k)
\end{cases}
\end{gather*}
где $(x_1, \dotsc, x_n)$ --- координаты в $\mathbb{R}^n$, $(t_1, \dotsc, t_k)$ --- координаты в $\mathbb{R}^k$,
при этом $(t_1, \dotsc, t_k) \in V$ тогда и только тогда, когда $(x_1, \dotsc, x_n)\in M$. При этом $\phi_1,\dotsc,\phi_n$ дифференцируемы по каждой переменной и матрица частных производных невырождена.
\begin{gather*}
\begin{pmatrix}
\frac{\partial \phi_1}{\partial t_1} & \frac{\partial \phi_2}{\partial t_1} & \cdots & \frac{\partial \phi_n}{\partial t_1} \\
\vdots & \vdots & \ddots & \vdots \\
\frac{\partial \phi_1}{\partial t_k} & \frac{\partial \phi_2}{\partial t_k} & \cdots & \frac{\partial \phi_n}{\partial t_k}
\end{pmatrix}
\end{gather*}
Ранг этой матрицы должен быть $k$ в любой точке $t \in V$ (то есть все строки должны быть линейно независимы).
\\\\
\textbf{Пример:}
\begin{gather*}
\begin{cases}
    x=t^2\\
    y=t^3
\end{cases}
\end{gather*}
Обе функции дифференцируемые, но в точке $t=0$ обе производные обращаются в ноль. Поэтому кривая не гладкая.

\section{Сформулируйте теорему о неявной функции. Допустим кривая $X \subseteq \mathbb{R}^2$ задана уравнением $f(x,y)=0$, и известно, что $\mathrm{grad}f(x_0,y_0)=(2;0)$. Какую из координат $x,y$ можно использовать в качестве локальной координаты на $X$ в окрестности точки $(x_0,y_0)$?}
\begin{theorem}
Пусть есть функция $F: \mathbb{R}^2\xrightarrow{} \mathbb{R}$, для которой выполнены условия:
\begin{enumerate}
\item $F$ определена и непрерывна в окрестности $(x_0,y_0)$ \item $F'_y(x_0,y_0)\not=0$ и $F'_y$ непрерывна в $(x_0,y_0)$ \item $F(x_0, y_0)=0$.
\end{enumerate}
Тогда найдётся окрестность $U_{\delta,\epsilon}(x_0,y_0)=\left\{(x,y)\left|\begin{array}{l}
     x \in (x_0-\delta, x_0 +\delta) \\
     y \in (y_0-\epsilon,y_0+\epsilon)
\end{array}\right.\right\}$ и непрерывная функция $f$ такая, что в $U_{\delta,\epsilon}(x_0,y_0)\quad$ $F(x,y)=0 \Leftrightarrow y = f(x)$ (то есть можно выразить $y$ от $x$ в данной окрестности при выполненных выше условиях).\\
Если кроме всех условий выше $F$ дифференцируема в $U_{\delta,\epsilon}(x_0,y_0)$, то $f$ дифференцируема в $U_{\delta}(x_0)$ и
\begin{gather*}
    f'(x_0)=-\frac{F'_x(x_0,y_0)}{F'_y(x_0,y_0)}
\end{gather*}
\end{theorem}
\textbf{Задача:} проверяем условия теоремы, производная по $x$ не равна нулю, а производная по $y$ равна. Значит в качестве координаты можно взять $y$, а $x$ --- нельзя. Обратите внимание, координата --- эта не та переменная, по которой дифференцируем.

\section{Сформулируйте общую теорему о неявном отображении. Допустим, кривая $X \subseteq \mathbb{R}^3$ задана уравнениями $f(x,y,z)=0,\; g(x,y,z)=0$, и известно, что $\mathrm{grad}f(x_0,y_0,z_0)=(2;0;0)$, $\mathrm{grad}f(x_0, y_0, z_0)=(0;1;3)$. Какие из координат $x,y,z$ можно использовать в качестве локальных координат на $X$ в окрестности точки $(x_0,y_0,z_0)$?}
\textbf{Обозначения:}
$x=(x_1, \dotsc, x_n)$, $y=(y_1, \dotsc y_m)$, $(x, y) = (x_1, \dotsc, x_n, y_1, \dotsc y_m)$.\\\
\textbf{Ещё обозначения:} если функции $g_1,\dotsc,g_s$ зависят от $t_1,\dotsc,t_r$, то
\begin{gather*}
    \frac{D(g_1,\dotsc,g_s)}{D(t_1,\dotsc,t_r)} =
    \begin{pmatrix}
        \frac{\partial g_1}{\partial t_1} & \frac{\partial g_1}{\partial t_2} & \cdots & \frac{\partial g_1}{\partial t_r} \\
        \vdots & \vdots & \ddots & \vdots \\
        \frac{\partial g_s}{\partial t_1} & \frac{\partial g_s}{\partial t_2} & \cdots & \frac{\partial g_s}{\partial t_r}
    \end{pmatrix}
\end{gather*}
(по строкам матрицы записаны градиенты (да, в 14 билете градиенты были записаны по столбцам, но так Айз давал на той лекции)).\\\\
Разрешим теперь $m$ уравнений относительно $m$ неизвестных.\\
\begin{theorem}
Пусть
\begin{enumerate}
    \item $F_1 (x,y),\dotsc,F_m(x,y)$ --- непрерывно дифференцируемы в окрестности точки $(x^{(0)}, y^{(0)})$ (здесь верхние индексы, чтобы не путать с координатами)
    \item $F_j(x^{(0)}, y^{(0)})=0 \quad \forall j=1,\dotsc,m$
    \item det$\frac{D(F_1,\dotsc,F_m)}{D(y_1,\dotsc,y_m)}|_{(x^{(0)}, y^{(0)})}\not=0$
\end{enumerate}
Тогда существует окрестность $U_\delta(x^{(0)}) \times U_\epsilon(y^{(0)})$ и набор дифференцируемых функций $f_1(x_1,\dotsc,x_n),\dots,f_m(x_1,\dotsc,x_m)$, таких что в этой окрестности
\begin{gather*}
    \{F_j(x,y)=0\}^m_{j=1} \Leftrightarrow \{y_j=f_j(x)\}^m_{j=1}
\end{gather*}
при этом $f_j(x^{(0)})=y_j^{(0)}$.\\
Более того,
\begin{gather*}
    \frac{D(f_1,\dotsc, f_m)}{D(x_1,\dotsc,x_n)}\Big|_{x^{(0)}}=-\Big(\frac{D(F_1,\dotsc,F_m)}{D(y_1,\dotsc,y_m)}\Big)^{-1}\Big|_{(x^{(0)}, y^{(0)})} \cdot \frac{D(F_1,\dotsc,F_m)}{D(x_1,\dotsc,x_n)}\Big|_{x^{(0)}}
\end{gather*}
\end{theorem}
\textbf{Задача:} Запишем матрицу
\begin{gather*}
\begin{pmatrix}
    2 & 0 & 0\\
    0 & 1 & 3
\end{pmatrix}
\end{gather*}
Видим, что линейно независимы первый и второй столбец, и первый и третьей. Значит координатой может быть $z$ или $y$. Обратите внимание, если матрица производных по $x$ и $y$ невырождена, то подходит как координата $z$.


\section{Дайте определение касательного вектора к подмножеству $X \subseteq \mathbb{R}^n$ в точке $A \in X$. Как устроено множество всех касательных векторов к гладкому подмногообразию в фиксированной точке?}
\begin{definition}
Пусть $x^{(0)} \in X \subseteq \mathbb{R}^n$. Построим какую-нибудь кривую, которая целиком лежит в $X$ и проходит через $x^{(0)}$. Пусть эта кривая задаётся параметрически ${x_i = \psi_i(s), s \in (-\epsilon, \epsilon)}$, и ${(\psi_1(s), \dotsc, \psi_n(s)) \in X \; \forall s\in (-\epsilon,\epsilon)}$, и $(\psi_1(0),\dotsc,\psi_n(0)) = x^{(0)}$. Тогда вектор $(\frac{d\psi_1}{ds}(0),\dotsc,\frac{d\psi_n}{ds}(0))$ называется \textit{касательным} к $X$ в точке $x^{(0)}$ (если такой вектор определён, конечно).
\end{definition}
\begin{remark}
Касательных векторов может быть бесконечно много, т. к. бесконечно много таких кривых.
\end{remark}
Пусть $X$ теперь --- гладкое $k$-мерное многообразие и $x_i=\phi_i(t_1,\dotsc,t_k)$ --- гладкие координаты в окрестности точки $x^{(0)} = \Phi(t^{(0)})$. Тогда множество касательных векторов в точке $x^{(0)}$ образует $k$-мерное векторное пространство (обозначается $T_{x^{(0)}}X$), линейно порождённое следующими векторами
\begin{gather*}
    \Big(\frac{\partial \phi_1}{\partial t_1}(t^{(0)}),\dotsc,\frac{\partial \phi_n}{\partial t_1}(t^{(0)})\Big)\\
    \vdots\\
    \Big(\frac{\partial \phi_1}{\partial t_k}(t^{(0)}),\dotsc,\frac{\partial \phi_n}{\partial t_k}(t^{(0)})\Big)
\end{gather*}
\begin{remark}
Эти векторы задают аффинное пространство, чтобы получить геометрическое касательное пространство, нужно сдвинуть $T_{x^{(0)}}X$ в точку $x^{(0)}$.
\end{remark}

\section{Допустим, что все точки множества $X \subset \mathbb{R}^n$ удовлетворяют уравнению $f(x)=0$. Докажите, что в любой точке $x^{(0)}\in X$ любой касательный вектор к $X$ перпендикулярен градиенту $\mathrm{grad}f(x^{(0)})$. Опишите касательное пространство к $k$-мерному подмногообразию $\mathbb{R}^n$, заданному системой неявных уравнений (без доказательства).}
Имеем $\forall x \in X \; f(x) = 0$. Тогда для любой кривой $\{x_i=\phi_i(s)\} \subset X$ имеем $f(\phi_1(s),\dotsc,\phi_n(s))=0$. продифференцируем это по $s$, получаем
\begin{gather*}
    \frac{\partial f}{\partial x_1} \cdot \frac{f\phi_1}{ds}+\dotsc + \frac{\partial f}{\partial x_n} \cdot \frac{f\phi_n}{ds} = 0\\
    <\mathrm{grad}f(x^{(0)}),\quad \text{касательный вектор к $X$ в точке $x^{(0)}$}>=0
\end{gather*}
Из того, что скалярное произведении равно нулю, следует, что градиент $f$ перпендикулярен касательному вектору к множеству $X$.\\\\
Касательное пространство --- ортогональное дополнение к линейной комбинации градиентов неявных уравнений.

\section{Необходимое и достаточное условия локального экстремума для функции нескольких переменных (без доказательства).}
\begin{definition}
	Точка $x^{(0)}$ функции $f$ называется \textit{стационарной}, когда $\frac{\partial f}{\partial x_i}(x^{(0)}) = 0 \quad\forall x \in [1;n]$.
\end{definition}
Необходимое условие:
\begin{theorem}
	Если $f(x^{(0)})$ - локальный экстремум, то $x^{(0)}$ --- стационарная.
\end{theorem}
\begin{definition}
    \textit{Матрицей Гессе} называется симметричная квадратичная форма
    \begin{gather*}
        \begin{pmatrix}
        \frac{\partial^2 f}{\partial x_1^2} & \frac{\partial^2 f}{\partial x_1 \partial x_2} & \cdots & \frac{\partial^2 f}{\partial x_1 \partial x_n}\\
        \vdots\\
        \frac{\partial^2 f}{\partial x_n \partial x_1} & \frac{\partial^2 f}{\partial x_n \partial x_2} & \cdots & \frac{\partial^2 f}{\partial x_n^2}
        \end{pmatrix}
    \end{gather*}
\end{definition}
\begin{definition}
    \textit{Положительно определённой} квадратичной формой называется такая, что все угловые миноры положительны.
\end{definition}
\begin{definition}
    \textit{Отрицательно определённой} квадратичной формой называется такая, что все угловые миноры отрицательны.
\end{definition}
\begin{theorem}
Теперь, пусть дана дважды дифференцируемая функция $f(x_1,\dotsc,x_n)$, пусть $x^{(0)}$ --- стационарная точка.
Тогда:
\begin{itemize}
    \item Если матрица Гессе положительна определена, то $x^{(0)}$ --- локальный минимум.
    \item Если матрица Гессе отрицательна определена, то $x^{(0)}$ --- локальный максимум.
    \item Если матрица Гессе имеет и положительные и отрицательные миноры, но при этом не вырождена, то $x^{(0)}$ --- не локальный экстремум.
    \item В остальных случаях $x^{(0)}$  может как являться локальным экстремумом, так и не являться.
\end{itemize}
\end{theorem}

\section{Дайте определение точки условного минимума}
\begin{definition}
    Точка $x^{(0)}$ называется \textit{строгим условным минимумом} функции $f$ подмножества $X \subset \mathbb{R}^n$, если $\forall x \in X \quad f(x) > f(x^{(0)})$.
\end{definition}
\begin{definition}
    Точка $x^{(0)}$ называется \textit{условным локальным минимумом} функции $f$ подмножества $X \subset \mathbb{R}^n$, если существует окрестность $U(x^{(0)})$, такая что $\forall x \in U(x^{(0)}) \cap X \quad f(x) > f(x^{(0)})$.
\end{definition}
\begin{remark}
Далее будем считать, что такое множество задаётся набором уравнений вида $\phi(x)=0$.
\end{remark}

\section{Сформулируйте теорему о множителях Лагранжа. Объясните идею доказательства в случае, если подмножество $X \subset \mathbb{R}^n$ является гладким многообразием.}
Пусть у нас есть задача вида
\begin{gather*}
    \begin{cases}
        f(x) \xrightarrow{} \mathrm{extr}\\
        \phi_1(x)=0\\
        \vdots\\
        \phi_m(x)=0
    \end{cases}\\
    x \in \mathbb{R}^n\\
    m < n
\end{gather*}
\begin{definition}
\textit{Функцией Лагранжа} называется
\begin{gather*}
    L(x,\lambda) = f(x) - \sum_{i=1}^m \lambda_i g_i(x)\\
    x \in \mathbb{R}^n\\
    \lambda \in \mathbb{R}^m
\end{gather*}
\end{definition}
$\lambda$ называют \textit{множителями Лагранжа}.\\
\begin{theorem}
\begin{sloppypar}
	Пусть $x^{(0)}$ --- точка условного локального экстремума в задаче выше, и пусть в окрестности точки $x^{(0)}$ $X$ --- гладкое многообразие. Тогда существуют такие $\lambda^{(0)}$, что точка ${(x^{(0)},\lambda^{(0)}) = (x^{(0)}_1,\dotsc, x^{(0)}_n, \lambda^{(0)}_1, \dotsc, \lambda^{(0)}_m) \in \mathbb{R}^{m+n}}$ является стационарной для $L(x, \lambda)$.
\end{sloppypar}
\end{theorem}
То есть
\begin{gather*}
\frac{\partial L}{\partial x_i}(x^{(0)},\lambda^{(0)}) = 0 \quad\forall i \in [1;n]\\
\frac{\partial L}{\partial \lambda_i}(x^{(0)},\lambda^{(0)}) = 0 \quad\forall i \in [1;m]\\
\end{gather*}
Второе в силу линейности по $\lambda$ эквивалентно $g_i(x^{(0)})=0$, что означает, что $x^{(0)} \in X$.\\
Посмотрим теперь на первое
\begin{gather*}
    \frac{\partial L}{\partial x_j} = \frac{\partial}{\partial x_j} (f(x) - \sum_{i=1}^m \lambda_i g_i(x)) = \frac{\partial  f}{\partial x_j} - \sum_{i=1}^m \lambda_i \frac{\partial g_i}{\partial x_j} = 0\\
    \begin{pmatrix}
        \frac{\partial f}{\partial x_1} \\
        \vdots\\
        \frac{\partial f}{\partial x_n}
    \end{pmatrix}
    - \lambda_1 \begin{pmatrix}
        \frac{\partial g_1}{\partial x_1} \\
        \vdots\\
        \frac{\partial g_1}{\partial x_n}
    \end{pmatrix}
    - \dotsc
    - \lambda_m \begin{pmatrix}
        \frac{\partial g_m}{\partial x_1} \\
        \vdots\\
        \frac{\partial g_m}{\partial x_n}
    \end{pmatrix} = \begin{pmatrix}
        0\\
        \vdots\\
        0
    \end{pmatrix}
\end{gather*}
Это значит, что
\begin{gather*}
    \mathrm{grad}f = \sum_{i=1}^m \lambda_i \mathrm{grad}g_i
\end{gather*}
Так как, все наши переходы были равносильными, нам осталось доказать, что найдутся такие $\lambda$, то есть, что $\mathrm{grad}f$ является линейной комбинацией $\mathrm{grad}g_i$ в данной точке.\\
Поскольку $X$ гладкая в точке $x^{(0)}$, будем предполагать, что $X$ удовлетворяет условию теоремы о неявном отображении, то есть градиенты $\mathrm{grad}g_i$ линейно независимы. Без ограничения общности будем считать $x^{(0)}\in X \subset \mathbb{R}^n$ --- точка условного локального минимума. Тогда, если возьмём какую-нибудь кривую $\{x_i=\phi(t)\} \subseteq X$, такую, что $\phi_i(0)=x_i^{(0)}$, то на ней это также будет точка локального минимума, запишем касательный вектор
\begin{gather*}
    u = (\frac{d\phi_1}{dt}(0),\dotsc,\frac{d\phi_n}{dt}(0)) \in T_{x^{(0)}}X
\end{gather*}
Функция $\alpha(t)=f(\phi_1(t),\dotsc,\phi_n(t))$ имеет в $t=0$ локальный минимум. По теореме Ферма $\frac{d \alpha}{dt}(0) = 0$. А это
\begin{gather*}
    \frac{\partial f}{\partial x_1}(x^{(0)})\cdot \frac{d \phi_1}{dt}(0)+
    \dotsc + \frac{\partial f}{\partial x_n}(x^{(0)})\cdot \frac{d \phi_n}{dt}(0) = <\mathrm{grad}f(x^{0}),u> = 0
\end{gather*}
Таким образом, градиент целевой функции в точки экстремума перпендикулярен любому касательному вектору $u \in T_{x^{(0)}}X$, то есть $\mathrm{grad}f(x^{(0)}) \perp T_{x^{(0)}}X$. А это значит, что этот градиент лежит в ортогональном дополнении
\begin{gather*}
    \mathrm{grad}f(x^{(0)}) \in (T_{x^{(0)}}X)^\perp = <\mathrm{grad}g_1(x^{(0)}),\dotsc,\mathrm{grad}g_m(x^{(0)})>
\end{gather*}
А раз $\mathrm{grad}f(x^{(0)})$ лежит в линейной оболочке $\mathrm{grad}g_i(x^{(0)})$, то он является их линейной комбинацией.
	\setcounter{section}{27}
	\section{Дайте определение элемента площади 2-мерной поверхности в $\mathbb{R}^3$ и поверхностного интеграла 1-го рода}

	Пусть имеется двумерная поверхность $\Omega \subseteq \mathbb{R}^3$ и у неё зафиксирована параметризация $\varphi:M \to \Omega, M \subseteq \mathbb{R}^2$. Будем обозначать координаты в $\mathbb{R}^3$ как $(x, y, z)$, а в $\mathbb{R}^2$ ---~ как $(u, v)$. Неформально говоря, элементом площади в точке поверхности называется площадь бесконечно малого параллелограмма со сторонами, направленными параллельно касательным векторам в этой точке. Можно провести аналогию с одномерными интегралами, где мы приближаем функцию с помощью ломаной с маленькими звеньями, и сказать, что мы приближаем поверхность маленькими чешуйками в форме параллелограммов. Запишем теперь формулу для элемента площади в точке $(u, v)$
	
	\[ \diff S = S(P(\varphi'_u(u, v), \varphi'_v(u, v))) \diff u \diff v; \]

	Здесь $\varphi'_u, \varphi'_v$ --- трёхмерные векторы (так как $\varphi$ имеет три координаты), именно они являются касательными в данной точке; $P$ ---~ параллелограмм, натянутый на векторы; $S$ ---~ площадь. Из линейной алгебры мы знаем, что площадь параллелограмма можно считать как корень из определителя матрицы Грама его сторон. Это даёт нам новую формулу для элемента площади.

	\[ \diff S = \sqrt{EG - F^2}\diff u \diff v; \]
	
	Здесь $E = \langle \varphi'_u, \varphi'_u \rangle = \| \varphi'_u \|^2, G = \langle \varphi'_v, \varphi'_v \rangle = \| \varphi'_v \|^2, F = \langle \varphi'_u, \varphi'_v \rangle$. 

	Теперь мы можем естественным образом определить поверхностный интеграл 1-го рода от функции $f:\mathbb{R}^3 \to \mathbb{R}$ по $\Omega$.

	\[ \iint\limits_\Omega f(x, y, z) \diff S := \iint\limits_M f(\varphi(u, v)) \sqrt{E(u, v)G(u, v) - F^2(u, v)} \diff u \diff v; \]

	Здесь мы опираемся на параметризацию при определении интеграла. Можно проверить, что при смене параметризации значение интеграла 1-го рода не изменится. 

	\section{Дайте определение элемента $k$-мерного объёма $k$-мерного многообразия в $\mathbb{R}^n$ и интеграла 1-го рода по $k$-мерному многообразию}

	Пусть имеется $k$-мерное многообразие $\Omega \subseteq \mathbb{R}^n$ и у него зафиксирована параметризация $\varphi: M \to \Omega, M \subseteq \mathbb{R}^k$. Будем обозаначать координаты в $\mathbb{R}^n$ как $x = (x_1, \ldots, x_n)$, а в $\mathbb{R}^k$ ---~ как $t = (t_1, \ldots, t_k)$. Аналогично предыдущему билету, определим элемент $k$-мерного объёма в точке $t$.

	\[ \diff \! \mathit{Vol}_k = S(P(\varphi'_{t_1}(t), \ldots, \varphi'_{t_k}(t))) \diff t_1 \ldots \diff t_k; \]

	Запишем теперь формулу для интеграла 1-го рода от функции $f:\mathbb{R}^n \to \mathbb{R}$ по $\Omega$.

	\[ \int\limits_\Omega f(x) \diff \! \mathit{Vol}_k := \int\limits_M f(\varphi(t))S(P(\varphi'_{t_1}(t), \ldots, \varphi'_{t_k}(t))) \diff t_1 \ldots \diff t_k;  \]

	Опять же, можно проверить, что интеграл 1-го рода не зависит от параметризации.

	\section{Объясните, что такое грассманово умножение, грассмановы переменные, грассмановы мономы}

	Пусть у нас имеется набор символов $a_1, \ldots, a_n$ ---~ грассмановых переменных и мы умеем брать их линейные комбинации. То есть, например, у нас есть отдельные элементы $a_2 - a_1, 0, -5a_3$ и т. п. Теперь мы хотим ввести новую операцию ---~ научиться умножать наши элементы друг на друга. Наше умножение будет обозначаться символом $\land$ и называться грассмановым умножением. Умножение будет удовлетворять всем стандартным требованиям, кроме коммутативности, которую мы заменим на более странное свойство 4:

	\begin{enumerate}
		\item $(x \land y) \land z = x \land (y \land z);$
		\item $(x + y) \land z = x \land z + y \land z;$
		\item $z \land (x + y) = z \land x + z \land y;$
		\item $a_i \land a_j = -a_j \land a_i;$
	\end{enumerate}

	Обратите внимание, пункты 1-3 относятся к любым элементам, а пункт 4 только к исходным $a_1, \ldots, a_n$. Простые следствия из свойств: $0 \land x = 0, a_i \land a_i = 0$. Для примера посчитаем <<квадрат>> элемента $a_1 \land a_2 + a_3$.
	
	\[ (a_1 \land a_2 + a_3) \land (a_1 \land a_2 + a_3) = a_1 \land a_2 \land (a_1 \land a_2 + a_3) + a_3 \land (a_1 \land a_2 + a_3) = 0 + a_1 \land a_2 \land a_3 + a_3 \land a_1 \land a_2 + 0 = \] \[ = a_1 \land a_2 \land a_3  - a_1 \land a_3 \land a_2 = a_1 \land a_2 \land a_3 + a_1 \land a_2 \land a_3 = 2a_1 \land a_2 \land a_3;\]

	\begin{definition}
		\textit{Грассмановым мономом} степени $k$ называется элемент вида $\alpha a_{i_1} \land \ldots \land a_{i_k}$, где $ i_1, \ldots i_k \in \{0, \ldots, n \}$, $\alpha$ ---~ некоторый коэффициент.
	\end{definition}

	Заметим, что если среди $i_1, \ldots i_k$ есть повторения, то моном равен нулю. Переменные в грассмановом мономе можно отсортировать, возможно, поменяв при этом знак. Точнее, при сортировке моном домножится на -1 в степени равной числу инверсий, то есть на знак перестановки.

	\section{Объясните, что такое дифференциальная форма ранга $k$, и как вычисляется интеграл (2-го рода) от $k$-формы $\omega$ по $k$-мерному многообразию $\Omega \subseteq \mathbb{R}^n$. Запишите вычислительную формулу для поверхностного интеграла 2-го рода}

	\begin{definition}
		\textit{Дифференциальной формой} ранга $k$ (или дифференциальной $k$-формой) на $M \subseteq \mathbb{R}^n$ называется выражение вида $\sum\limits_{\{i_1, \ldots, i_k\} \subseteq \{1, \ldots, n\}} f_{i_1 \ldots i_k}(x)\diff x_{i_1} \land \ldots \land \diff x_{i_k}$, где $f_{i_1\ldots i_k}$ ---~ некоторые дифференцируемые\footnote{Часто ограничиваются гладкими функциями.} функции $f_{i_1\ldots i_k}:M \to \mathbb{R}$.
	\end{definition}

	Если вам очень понравился предыдущий билет, можно сказать, что это сумма грассмановых мономов степени $k$ от переменных $\diff x_1, \ldots \diff x_n$ с дифференцируемыми функциями в качестве коэффициентов. Можно считать, что среди чисел $i_1, \ldots, i_k$ нет повторений, так как мономы с повторениями всё равно зануляются.

	Пусть имеются $k$-мерное многообразие $\Omega \subseteq \mathbb{R}^n$ с параметризацией $\varphi: M \to \Omega, M \subseteq \mathbb{R}^k$ и дифференциальная $k$-форма $\omega = \sum\limits_{\{i_1, \ldots, i_k\} \subseteq \{1, \ldots, n\}} f_{i_1 \ldots i_k}(x)\diff x_{i_1} \land \ldots \land \diff x_{i_k}$ на $\Omega$. Определим интеграл (2-го рода) $\omega$ по $\Omega$.

	\[ \int\limits_\Omega \omega := \int\limits_M \sum_{\{i_1, \ldots, i_k\} \subseteq \{1, \ldots, n\}} f_{i_1 \ldots i_k} (\varphi(t)) \diff \varphi_{i_1} \land \ldots \land \diff \varphi_{i_k}; \]

	Поясним, что творится в этой формуле. Во-первых, $\varphi_i : M \to \mathbb{R}$ ---~ это функция, соответствующая $i$-й координате $\varphi$. Во-вторых, $\diff \varphi_i$ ---~ это привычный дифференциал функции нескольких переменных, но теперь мы говорим, что это линейная комбинация грассмановых переменных $\diff t_1, \ldots, \diff t_k$. Когда мы грассманово перемножим эти дифференциалы, у нас останется выражение вида $f(t) \diff t_1 \land \ldots  \land \diff t_k$. Это так, ведь в любом слагаемом результата будут перемножаться $k$ переменных, одинаковые занулятся, останутся только слагаемые с различными, возможно, не в том порядке. Но мы можем привести порядок к правильному. После этих преобразований мы считаем интеграл как обычный кратный интеграл.
\[\int\limits_M f \diff t_1 \land \ldots \land \diff t_k = \int\limits_M f \diff t_1 \ldots \diff t_k;  \]  

	Для случая $k = 2$ это всё можно записать в следующую формулу.

	\[\iint\limits_\Omega P\diff y \land \diff z + Q \diff z \land \diff x + R \diff x \land \diff y = \iint\limits_M \begin{vmatrix} P & Q & R \\ \frac{\partial \varphi_1}{\partial u} & \frac{\partial \varphi_2}{\partial u} & \frac{\partial \varphi_3}{\partial u} \\ \frac{\partial \varphi_1}{\partial v} & \frac{\partial \varphi_2}{\partial v} & \frac{\partial \varphi_3}{\partial v} \end{vmatrix} \diff u \diff v;\]

	Обратите внимание, при $Q$ стоит $\diff z \land \diff x$, а не $\diff x \land \diff z$. 

	\section{Что такое ориентация $k$-мерного многообразия? Как изменится интеграл 2-го рода от дифференциальной формы при смене ориентации многообразия (б. д.)?}

	Пусть $\Omega \subseteq \mathbb{R}^n$ ---~ $k$-мерное связное\footnote{Напомним, многообразие называется гладким, если любые две его точки можно соединить проходяще по нему непрерывной кривой.} многообразие, и у него имеются две параметризации $\varphi: M \to \Omega, \psi: N \to \Omega; M, N \subseteq \mathbb{R}^k$. Предположим, что функция замены координат $c = \varphi^{-1} \circ \psi$ биективна и непрерывно дифференцируема.

	\[ \begin{tikzcd}
		M \arrow[swap]{rd}{\varphi} &  &\arrow[swap]{ll}{c} N \arrow{ld}{\psi} \\
		& \Omega
	\end{tikzcd} \]

	Посмотрим на якобиан $J(c)$. Если бы где-то он был равен нулю, в окрестности этой точки $c$ была бы необратима. Значит он не равен нулю нигде. Поскольку $J(c)$ непрерывен и $\Omega$ связно, из этого следует, что он имеет постоянный знак. Тогда если он положителен, будем говорить, что $\varphi$ и $\psi$ задают одну и ту же ориентацию, а если отрицателен ---~ то разные. Таким образом мы определяем ориентацию как отношение эквивалентности с двумя классами на параметризациях многообразия.

	Ориентация задаёт ориентацию на любом касательном пространстве $T_{x}\Omega$ как на векторном пространстве. Если мы назвали ориентацию некоторой параметризации положительной, то назовём положительным базис $T_x \Omega$, полученный из её производных.

	При смене параметризации на имеющую противоположную ориентацию интеграл 2-го рода меняет знак. 

	\section{Дайте определение согласованных ориентаций многообразия и его границы. Дайте определение дифференциала от $k$-формы. Запишите общую формулу Стокса.}
	Будем обозначать границу многообразия $\Omega$ как $\partial \Omega$. Заметим, что если у $k$-мерного многообразия есть граница, то она имеет размерность $k - 1$.

	\begin{definition}
		Будем говорить, что ориентации $\Omega$ и $\partial \Omega$ согласованы, если для любой точки $x \in \partial \Omega$ для любого положительного базиса $v_1, \ldots v_{k - 1}$ в $T_x \partial \Omega$, базис $v_1, \ldots, v_{k - 1}, \vec{n}$ положителен в $T_x \Omega$, где $\vec{n}$ ---~ это вектор в $T_x \Omega$, перпендикулярный $T_x \partial \Omega$ и смотрящий наружу\footnote{Это можно формализовать, например, как отрицательное скалярное произведение с любым вектором, соединяющим $x$ и точку из окрестности $x$ из $\Omega$. Но лектор это никак не формализовал.} $\Omega$.
	\end{definition}

	Мы привыкли, что дифференциал суммы равен сумме дифференциалов. Поэтому для определения дифференциала от дифференциальной формы достаточно определить дифференциал от грассманова монома.

	\[ \diff(f \diff x_{i_1} \land \ldots \land \diff x_{i_k}) := \diff f \land \diff x_{i_1} \land \ldots \land \diff x_{i_k}; \]

	\begin{remark}
		Дифференциал $k$-формы является $(k + 1)$-формой.
	\end{remark}

	Для примера посчитаем дифференциал от дифференциала некоторой функции $f:\mathbb{R}^n \to \mathbb{R}$.

	\[\diff (\diff f) = \diff \left(\sum_{i = 1}^n \frac{\partial f}{\partial x_i}\diff x_i \right) = \sum_{i = 1}^n \sum_{j = 1}^n \frac{\partial^2 f}{\partial x_i \partial x_j} \diff x_j \land \diff x_i; \]

	При этом слагаемые вида $\frac{\partial^2 f}{\partial x_i \partial x_i} \diff x_i \land \diff x_i$ сразу зануляются, а слагаемые вида $\frac{\partial^2 f}{\partial x_i \partial x_j} \diff x_j \land \diff x_i$ сократятся с $\frac{\partial^2 f}{\partial x_j \partial x_i} \diff x_i \land \diff x_j$. Значит $\diff(\diff f) = 0$.

	Пусть теперь $\Omega \subseteq \mathbb{R}^n$ ---~ $k$-мерное многообразие с согласованными ориентациями на самом многообразии и на границе, а $\omega$ ---~ дифференциальная $(k - 1)$-форма на $\Omega$. Тогда верна (общая) формула Стокса:

	\[ \int\limits_{\partial \Omega} \omega = \int\limits_\Omega \diff \omega; \] 

	\section{Выведите из общей формулы Стокса частные случаи: формулу Ньютона-Лейбница, формулу Грина, формулу Гаусса-Остроградского.}

	\subsection*{Формула Ньютона-Лейбница, $n = k = 1$}

	Пусть наше многообразие это отрезок на прямой $[a; b]$. Его границей будет множество из двух точек $\{a, b\}$. Заметим, что точка ---~ это нульмерное многообразие и по нему можно интегрировать 0-формы (то есть просто функции). При чём этот интеграл будет с точностью до знака (знак как всегда определяется ориентацией) равен значению функции в точке. Если на отрезке мы берём стандартную ориентацию <<слева направо>>, то для границы это будет означать взятие $b$ с плюсом и $a$ с минусом. Итак, формула Стокса принимает следующий вид:
	
	\[F(b) - F(a) = \int\limits_a^b \diff F; \]
	
	Перепишем в более привычную запись.

	\[\int\limits_a^b F'(x) \diff x = F(b) - F(a); \]

	\subsection*{Формула Грина, $n = k = 2$}

	Дифференциальная 1-форма в $\mathbb{R}^2$ имеет вид $P\diff x + Q \diff y$. Посчитаем её дифференциал

	\[\diff (P \diff x + Q \diff y) = \left(\frac{\partial P}{\partial x}\diff x + \frac{\partial P}{\partial y}\diff y \right) \land \diff x +\left(\frac{\partial Q}{\partial x}\diff x + \frac{\partial Q}{\partial y}\diff y \right) \land \diff y = \frac{\partial P}{\partial y} \diff y \land \diff x + \frac{\partial Q}{\partial x} \diff x \land \diff y = \] \[ = \left(\frac{\partial Q}{\partial x} - \frac{\partial P}{\partial y}\right) \diff x \land \diff y;  \]

	Формула Стокса принимает следующий вид:

	\[\int\limits_{\partial U} P\diff x + Q\diff y = \iint\limits_{U} \left(\frac{\partial Q}{\partial x} - \frac{\partial P}{\partial y}\right)\diff x \diff y; \]

	Здесь условие согласованности ориентации можно сформулировать как <<при обходе $\partial U$ по заданной параметризации $U$ всегда находится слева>>.

	\subsection*{Формула Гаусса-Остроградского, $n = k = 3$}

	Дифференциальная 2-форма в $\mathbb{R}^3$ имеет вид $P\diff y \land \diff x + Q \diff z \land \diff x + R \diff x \land \diff y$. Посчитаем её дифференциал.

	\[\diff (P\diff y \land \diff z + Q \diff z \land \diff x + R \diff x \land \diff y) = \frac{\partial P}{\partial x} \diff x \land \diff y \land \diff z + \frac{\partial Q}{\partial y}\diff y \land \diff z \land \diff x + \frac{\partial R}{\partial z} \diff z \land \diff x \land \diff y  = \] \[ =\left( \frac{\partial P}{\partial x} + \frac{\partial Q}{\partial y} + \frac{\partial R}{\partial z} \right) \diff x \land \diff y \land \diff z; \]

	Формула Стокса принимает следующий вид:

	\[\iint\limits_{\partial V} P\diff y \land \diff x + Q \diff z \land \diff x + R \diff x \land \diff y = \iiint\limits_{V}\left( \frac{\partial P}{\partial x} + \frac{\partial Q}{\partial y} + \frac{\partial R}{\partial z} \right) \diff x\diff y\diff z; \]
\end{document}