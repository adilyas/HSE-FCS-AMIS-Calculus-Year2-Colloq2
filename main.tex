\documentclass{article}
\usepackage{packages}

\title{Коллоквиум по Математическому анализу-2, семестр 2}
\author{Виноградова Дарья, Залялов Александр, Миронов Алексей, Стрельцов Артём, Т}
\date{}

\begin{document}

	\maketitle

	\tableofcontents

	\clearpage

	\setcounter{section}{27}
	\section{Дайте определение элемента площади 2-мерной поверхности в $\mathbb{R}^3$ и поверхностного интеграла 1-го рода}

	Пусть имеется двумерная поверхность $\Omega \subseteq \mathbb{R}^3$ и у неё зафиксирована параметризация $\varphi:M \to \Omega, M \subseteq \mathbb{R}^2$. Будем обозначать координаты в $\mathbb{R}^3$ как $(x, y, z)$, а в $\mathbb{R}^2$ ---~ как $(u, v)$. Неформально говоря, элементом площади в точке поверхности называется площадь бесконечно малого параллелограмма со сторонами, направленными параллельно касательным векторам в этой точке. Можно провести аналогию с одномерными интегралами, где мы приближаем функцию с помощью ломаной с маленькими звеньями, и сказать, что мы приближаем поверхность маленькими чешуйками в форме параллелограммов. Запишем теперь формулу для элемента площади в точке $(u, v)$
	
	\[ \diff S = S(P(\varphi'_u(u, v), \varphi'_v(u, v))) \diff u \diff v; \]

	Здесь $\varphi'_u, \varphi'_v$ --- трёхмерные векторы (так как $\varphi$ имеет три координаты), именно они являются касательными в данной точке; $P$ ---~ параллелограмм, натянутый на векторы; $S$ ---~ площадь. Из линейной алгебры мы знаем, что площадь параллелограмма можно считать как корень из определителя матрицы Грама его сторон. Это даёт нам новую формулу для элемента площади.

	\[ \diff S = \sqrt{EG - F^2}\diff u \diff v; \]
	
	Здесь $E = \langle \varphi'_u, \varphi'_u \rangle = \| \varphi'_u \|^2, G = \langle \varphi'_v, \varphi'_v \rangle = \| \varphi'_v \|^2, F = \langle \varphi'_u, \varphi'_v \rangle$. 

	Теперь мы можем естественным образом определить поверхностный интеграл 1-го рода от функции $f:\mathbb{R}^3 \to \mathbb{R}$ по $\Omega$.

	\[ \iint\limits_\Omega f(x, y, z) \diff S := \iint\limits_M f(\varphi(u, v)) \sqrt{E(u, v)G(u, v) - F^2(u, v)} \diff u \diff v; \]

	Здесь мы опираемся на параметризацию при определении интеграла. Можно проверить, что при смене параметризации значение интеграла 1-го рода не изменится. 

	\section{Дайте определение элемента $k$-мерного объёма $k$-мерного многообразия в $\mathbb{R}^n$ и интеграла 1-го рода по $k$-мерному многообразию}

	Пусть имеется $k$-мерное многообразие $\Omega \subseteq \mathbb{R}^n$ и у него зафиксирована параметризация $\varphi: M \to \Omega, M \subseteq \mathbb{R}^k$. Будем обозаначать координаты в $\mathbb{R}^n$ как $x = (x_1, \ldots, x_n)$, а в $\mathbb{R}^k$ ---~ как $t = (t_1, \ldots, t_k)$. Аналогично предыдущему билету, определим элемент $k$-мерного объёма в точке $t$.

	\[ \diff \! \mathit{Vol}_k = S(P(\varphi'_{t_1}(t), \ldots, \varphi'_{t_k}(t))) \diff t_1 \ldots \diff t_k; \]

	Запишем теперь формулу для интеграла 1-го рода от функции $f:\mathbb{R}^n \to \mathbb{R}$ по $\Omega$.

	\[ \int\limits_\Omega f(x) \diff \! \mathit{Vol}_k = \int\limits_M f(\varphi(t))S(P(\varphi'_{t_1}(t), \ldots, \varphi'_{t_k}(t))) \diff t_1 \ldots \diff t_k;  \]

	Опять же, можно проверить, что интеграл 1-го рода не зависит от параметризации.

	\section{Объясните, что такое грассманово умножение, грассмановы переменные, грассмановы мономы}

	Пусть у нас имеется набор символов $a_1, \ldots, a_n$ ---~ грассмановых переменных и мы умеем брать их линейные комбинации. То есть, например, у нас есть отдельные элементы $a_2 - a_1, 0, -5a_3$ и т. п. Теперь мы хотим ввести новую операцию ---~ научиться умножать наши элементы друг на друга. Наше умножение будет обозначаться символом $\land$ и называться грассмановым умножением. Умножение будет удовлетворять всем стандартным требованиям, кроме коммутативности, которую мы заменим на более странное свойство 4:

	\begin{enumerate}
		\item $(x \land y) \land z = x \land (y \land z);$
		\item $(x + y) \land z = x \land z + y \land z;$
		\item $z \land (x + y) = z \land x + z \land y;$
		\item $a_i \land a_j = -a_j \land a_i;$
	\end{enumerate}

	Обратите внимание, пункты 1-3 относятся к любым элементам, а пункт 4 только к исходным $a_1, \ldots, a_n$. Простые следствия из свойств: $0 \land x = 0, a_i \land a_i = 0$. Для примера посчитаем <<квадрат>> элемента $a_1 \land a_2 + a_3$.
	
	\[ (a_1 \land a_2 + a_3) \land (a_1 \land a_2 + a_3) = a_1 \land a_2 \land (a_1 \land a_2 + a_3) + a_3 \land (a_1 \land a_2 + a_3) = 0 + a_1 \land a_2 \land a_3 + a_3 \land a_1 \land a_2 + 0 = \] \[ = a_1 \land a_2 \land a_3  - a_1 \land a_3 \land a_2 = a_1 \land a_2 \land a_3 + a_1 \land a_2 \land a_3 = 2a_1 \land a_2 \land a_3;\]

	\begin{definition}
		\textit{Грассмановым мономом} называется элемент вида $\alpha a_{i_1} \land \ldots \land a_{i_k}$, где $\alpha \in R, i_1, \ldots i_k \in \{0, \ldots, n \}$.
	\end{definition}

	Заметим, что если среди $i_1, \ldots i_k$ есть повторения, то моном равен нулю. Переменные в грассмановом мономе можно отсортировать, возможно, поменяв при этом знак. Точнее, при сортировке моном домножится на -1 в степени равной числу инверсий, то есть на знак перестановки.

\end{document}