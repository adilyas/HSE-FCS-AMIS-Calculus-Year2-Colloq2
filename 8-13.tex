\usepackage{commath}

\makeatletter
\newcommand*{\toto}[2]{\mathrel{
  \settowidth{\@tempdima}{$\scriptstyle#1$}
  \settowidth{\@tempdimb}{$\scriptstyle#2$}
  \ifdim\@tempdimb>\@tempdima \@tempdima=\@tempdimb\fi
  \mathop{\vcenter{
    \offinterlineskip\ialign{\hbox to\dimexpr\@tempdima+1em{##}\cr
    \rightarrowfill\cr\noalign{\kern.5ex}
    \rightarrowfill\cr}}}\limits^{\!#1}_{\!#2}}}
\makeatother


\title{Билеты 8-13}
\author{}
\date{}
\begin{document}
\maketitle

\setcounter{section}{7}
\section{Сформулируйте (б.д.) лемму Римана. Какова связь между порядком дифференцируемости $2\pi$-периодической функции и асимптотикой ее коэффицинтов Фурье? Ответ поясните.}
\begin{lemma} (Риман).
    Пусть $f$ абсолютно интегрируема на промежутке $(a, b)$. Тогда
    $$
        \lim\limits_{\omega\to\infty}\int_{a}^{b}f(x)\sin(\omega x)dx = 0
    $$
    $$
        \lim\limits_{\omega\to\infty}\int_{a}^{b}f(x)\cos(\omega x)dx = 0
    $$
\end{lemma}

\begin{corollary}
    Если $f$ интегрируемая и $2\pi$-периодическая, то ее коэффициенты $a_n, b_n$ ряда Фурье стремятся к 0ю
\end{corollary}

\begin{corollary}
    Пусть $f$ -- $2\pi$-периодическая и имеет $s - 1$ производную, а $f^{(s - 1)}$ -- кусочно-гладкая. Тогда ряд Фурье для $f^{(s)}$ получается $s$-кратным почленным дифференцированием ряда Фурье для $f$. При этом, $a_n, b_n = \overline{o}\left(\frac{1}{k^s}\right)$ при $k\to\infty$
\end{corollary}
Почему второе следствие верно? Интуитивно: чтобы выполнялась лемма Римама, требуется, чтобы коэффициенты стремились к 0. При дифференцировании $s - 1$ раз у нас столько же раз <<вылезет>> $k$ из косинуса и синуса, поэтому, чтобы все еще выполнялась сходимость к 0, нужно, чтобы $a_n, b_n = \overline{o}\left(\frac{1}{k^s}\right)$.

\section{Дайте определение преобразования Фурье для функции $\mathbb{R}\to\mathbb{C}$, приведите пример вычисления преобразования Фурье. Пусть $f\in L_1(\mathbb{R})$. Сформулируйте основную теорему об образе преобразования Фурье.}
\begin{definition}
    Пусть $f\colon\mathbb{R}\to\mathbb{C}$. Если выражение $$\hat{f}(y) = F[f](y) = v.p.\int_{-\infty}^{+\infty}f(x)e^{ixy}dx$$ определено для всех $y\in\mathbb{R}$, то $\hat{f}\colon\mathbb{R}\to\mathbb{C}$ называется преобразованием Фурье.
\end{definition}

Например, пусть $f(x) = I_{[-1, 1]}(x)$. Тогда
$$
    \hat{f}(y) = \int_{-\infty}^{+\infty}I_{[-1, 1]}(x)e^{ixy}dx = \int_{-1}^{1}e^{ixy}dx = \int_{-1}^{1}\cos(xy)dx + i\int_{-1}^{1}\sin(xy)dx = \frac{\sin(xy)}{y}\Big|_{-1}^{1} = 2\frac{\sin y}{y}
$$

\begin{definition}
    $L_1(\mathbb{R})$ -- пространство функций, абслолютно интегрируемых на $\mathbb{R}$.
\end{definition}

\begin{theorem} (Основная теорема об образе Фурье)
    Если $f\in L_1(\mathbb{R})$, то
    \begin{enumerate}
        \item Функция $\hat{f}$ корректно определена.
        \item Функция $\hat{f}$ непрерывна.
        \item $\lim\limits_{y\to\pm\infty}\hat{f}(y) = 0$
    \end{enumerate}
\end{theorem}

\section{Что вы можете сказать о функции $\hat{f}$, если (1) $f$ четная и вещественнозначная (2) $f$ нечетная и вещественнозначная.}
\begin{lemma}
    \begin{enumerate}
        \item Если $f$ -- четная, то $F^{-1}[f] = \frac{1}{2\pi}F[f]$
        \item Если $f$ -- нечетная, то $F^{-1}[f] = -\frac{1}{2\pi}F[f]$
    \end{enumerate}
\end{lemma}

\begin{proof}
    \begin{enumerate}
        \item $F[f](y) = \int_{-\infty}^{\infty}f(x)e^{ixy}dx = \int_{-\infty}^{\infty}f(-x)e^{ixy}dx = [\textrm{Замена $x$ на $-x$}] = \int_{-\infty}^{\infty}f(x)e^{-ixy}d(-x) = \int_{\infty}^{-\infty}f(x)e^{-ixy}d(x) = 2\pi F^{-1}[f]$
        \item Доказывается абсолютно аналогично. При замене $x$ на $-x$ просто еще и поменяется знак у интеграла.
    \end{enumerate}
\end{proof}

\section{Дайте определение интеграла Фурье и обратного преобразования Фурье. Сформулируйте и докажите теорему о свертке. Чему равно преобразование Фурье от произведения двух функций?}
\begin{definition}
    Интегралом Фурье от функции $f\colon \mathbb{R}\to\mathbb{C}$ в точке $x$ называется выражение $$\frac{1}{2\pi} v.p.\int_{-\infty}^{\infty}\hat{f}(y)e^{-ixy}dy$$
\end{definition}

\begin{definition}
    Пусть $f$ дифференцируемая функция, тогда преобразование
        $$\hat{f}(y) \mapsto f(x) = \frac{1}{2\pi} v.p.\int_{-\infty}^{\infty}\hat{f}(y)e^{-ixy}dy$$
    называется обратным преобразованием Фурье.
\end{definition}

\begin{theorem} (О свертке)
    Если $f, g\in L_1(\mathbb{R})$, то $F(f\star g) = F[f]\cdot f[g]$.
\end{theorem}
\begin{proof}
    $$
    F[f\star g](y) = \int_{-\infty}^{+\infty}(f - g)(x)e^{ixy}dx = \int_{-\infty}^{+\infty}\left(\int_{-\infty}^{+\infty}f(s)g(x - s)ds\right)e^{ixy}dx
    $$
    Благодаря абсолютной интегрируемости, можем поменять порядки интегрирования, а следовательно, просто в виде кратного:
    $$
    \iint_{\mathbb{R}^2} f(s)g(x - s)e^{ixy}dxds
    =
    \begin{bmatrix}
        u = s \\
        v = x - s \\
        dxds = dudv
    \end{bmatrix}
    =
    \int_{-\infty}^{+\infty} f(u)e^{iuy}du\cdot \int_{-\infty}^{+\infty} g(v)e^{ivy}dv = F[f] \cdot F[g].
    $$
    Аналогично можно доказать для обратного преобразования.
\end{proof}

\begin{corollary}
    $F[f\cdot g] = \frac{1}{2\pi}F[f]\star F[g]$
\end{corollary}

\section{Выведите формулу для производной $\hat{f}(y)$. Какова связь дифференцируемости функции $f(x)$ и асимптотики ее преоразования Фурье $\hat{f}(y)$ при $y\to\infty$}

\begin{statement}
    Пусть $f(x)$ такова, что $f(x)(1 + \abs{x})^k\in L_1(\mathbb{R})$. Тогда $\hat{f} = F[f]$ дифференцируема $k$ раз, причем производные $\hat{f}$ можно вычислить дифференцированием под знаком интеграла.
\end{statement}

\begin{proof}
    Это следует из теорем прошлого семестра (что можно дифференцировать по параметру, в нашем случае по $y$, под знаком интеграла, если есть равномерная сходимость по $x$).
    $$
    F[f](y) = v.p.\int_{-\infty}^{+\infty}f(x)e^{ixy}dx
    $$
    $\forall\; m \le k$ верно
    $$
    \abs{f(x)e^{ixy}(ix)^m} \le \abs{f(x)}(1 + \abs{x})^m
    $$
    В левой части стоит просто продифференцированное $m$ раз подынтегральное выражение в преобразовании Фурье. Неравенство следует из ограниченности $\abs{e^{ixy}\cdot i^m} \le 1$. В правой части стоит функция, интегрируемая на $\mathbb{R}$ (по условию).\\
    При этом левая часть сходится равномерно по $y$. Значит, можем пользоваться теоремой из прошлого семестра:
    $$
        \frac{d^m}{dy^m}\hat{f}(y) = \int_{-\infty}^{+\infty}f(x)e^{ixy}(ix)^mdx
    $$
    Отсюда же получаем формулу:
    $$
    \frac{d^m}{dy^m}F[f](y) = F[(ix)^m f(x)]
    $$
\end{proof}

\begin{statement}
    Пусть $f$ дифференцируема $k$ раз во всех точках, причем $f^{(m)}\in L_1(\mathbb{R})$ при $m = 0, \ldots, k$ и $f^{(m)}(x)\to 0$ при $x\to\infty$ и $m = 0, \ldots, k$, тогда $\hat{f}(y) = \overline{o}\left(\frac{1}{\abs{y}^k}\right)$ при $t\to\infty$
\end{statement}

\begin{proof}
    Запишем $F[f^(k)]$:
    $$
    \int_{-\infty}^{+\infty}f^{(k)}(x)e^{ixy}dx = \int_{-\infty}^{+\infty}e^{ixy}df^(k - 1) = e^{ixy}f^{(k - 1)}(x)\Big|_{-\infty}^{+\infty} - \int_{-\infty}^{+\infty}f^{(k - 1)}(x)\cdot (iy)e^{ixy}dx =
    \begin{bmatrix}
    \textrm{Далее аналогично.} \\
    \textrm{Также пользуемся тем, что }\\
    \lim\limits_{x\to\infty}f^{(m)} = 0\;\;\forall m = 0, \ldots, k
    \end{bmatrix}
    $$
    $$
    =
    \int_{-\infty}^{+\infty}f(x)e^{ixy}(-1)^m (iy)^m dx
    =
    (-iy)^k F[f](y) = (-iy)^k \hat{f}(y)
    $$
    Также мы знаем, что если функция абсолютно интегрируемая, то ее Фурье-образ стремится к 0 (см. билет 9). Тогда
    $$f^{(k)}\in L_1(\mathbb{R}) \Rightarrow \abs{F[f^(k)](y)} = y^k F[f](y) \to 0 \Rightarrow \hat{f}(y) = \overline{o}\left(\frac{1}{\abs{y^k}}\right)$$
    
    P.S. И также, абсолютно аналогично с предыдущим пунктом выведем:
    $$
        F\left[\frac{d^m}{dx^m}f(x)\right] = (-iy)^m F[f]
    $$
\end{proof}

\section{Сформулируйте и докажите равенство Планшереля для преобразования Фурье.}
\begin{theorem} (Равенство Планшереля)
    Пусть $f, g\in L_1(\mathbb{R})$ ($f, g\colon \mathbb{R}\to\mathbb{C}$) и, кроме того, $f'' \in L_1(\mathbb{R})$, а также $f(x), f'(x)\to 0$ при $x\to\infty$. Тогда
    $$
    \int_{-\infty}^{+\infty}f(x)\overline{g(x)}dx = \frac{1}{2\pi}\int_{-\infty}^{+\infty}\hat{f}(y)\overline{\hat{g}(y)}dy
    $$
\end{theorem}

\begin{proof}
    Функция дифференцируема, поэтому для нее работает формула обращения Фурье:
    $$
    f(x) = \frac{1}{2\pi} v.p.\int_{-\infty}^{+\infty}\hat{f}(y)e^{-ixy}dy
    $$
    Тогда
    $$
    \int_{-\infty}^{+\infty}f(x)\overline{g(x)}dx = \int_{-\infty}^{+\infty}\frac{1}{2\pi}\left(\int_{-\infty}^{+\infty}\hat{f}(y)e^{ixy}dy\right)\overline{g(x)}dx
    =
    \frac{1}{2\pi}\int_{-\infty}^{+\infty}\hat{f}(y)\left(\int_{-\infty}^{+\infty}\overline{g(x)}\cdot \overline{e^{ixy}}dx\right)dy
    $$
    $$
    =
    \frac{1}{2\pi}\int_{-\infty}^{+\infty}\hat{f}(y)\overline{\left(\int_{-\infty}^{+\infty}g(x)e^{ixy}dx\right)}dy = \frac{1}{2\pi}\int_{-\infty}^{+\infty}\hat{f}(y)\overline{\hat{g}(y)}dy
    $$
\end{proof}
\end{document}
